% This syllabus template was created by:
% Brian R. Hall
% Assistant Professor, Champlain College
% www.brianrhall.net

% Document settings
\documentclass[11pt]{article}
\usepackage[margin=1in]{geometry}
\usepackage[pdftex]{graphicx}
\usepackage[sc]{mathpazo}
\pagestyle{fancy}
\setlength\parindent{0pt}

\begin{document}

% Course information
\begin{tabular}{ l l }
  \multirow{}{} & \LARGE LNG 230/SPV 221 \\\\
  & \LARGE Lehman College, CUNY \\\
  & \LARGE \textbf{Language Acquisition} \\\\
  & \LARGE MON WES, 8:00-9:15, Speech/Thr 202 \\\\
\end{tabular}
\vspace{10mm}

% Professor information
\begin{tabular}{ l l }
  \multirow{}{} & \large Instructor: Amy (Xiaomeng) Ma Graduate Teaching Fellow \\
  & \large Email: Xiaomeng.Ma@lehman.cuny.edu\\
  & \large Office Hours: by appointment \\
\end{tabular}
\vspace{5mm}

% Course details
\textbf {\large \\ Course Description:} A survey of psychological growth and development during childhood, adolescence, young adulthood, middle age, and old age. The emphasis will be placed on developmental tasks as distinguishing features of successive life stages. Patterns of intellectual growth, psychological growth under different social-cultural conditions, personality, and social development will be considered. \\
\textbf {Prerequisite(s):}  PSY 100, ENG 111, COR 100.

\textbf {\large Text(s):} \emph{How Children Develop}, 4\textsuperscript{th} Edition

\textbf {Author(s):} Siegler, R.S., DeLoache, J.S., Eisenberg, N.,Saffran, J. \\  
\textbf {ISBN-13:} 978-0000000000 \\

\textbf {\large Course Objectives:} \\
At the completion of this course, you:
\begin{enumerate} \itemsep-0.4em
  \item will learn how people behave at various points in development and how their behavior changes from infancy to adolescence and beyond.
  \item will learn some of the ways that psychologists conceptualize development and understand the strength and scope of several major theories. 
  \item will become versed in developmental methods and be able to think about developing behavior using the tools of the trade. 
  \item will be able to relate the facts, theories, and methods of developmental psychology to everyday problems and real world concerns. 
\end{enumerate}



% I recommend using \newpage here if necessary

\textbf {\large Letter Grade Distribution:} \\\\
\hspace*{40mm}
\begin{tabular}{ l l | l l }
\textgreater= 93.00 & A & 73.00 - 76.99 & C \\
90.00 - 92.99 & A-  & 70.00 - 72.99 & C- \\
87.00 - 89.99 & B+  & 67.00 - 69.99 & D+ \\
83.00 - 86.99 & B  & 63.00 - 66.99 & D \\
80.00 - 82.99 & B-  & 60.00 - 62.99 & D- \\
77.00 - 79.99 & C+  & \textless= 59.99 & F \\
\end{tabular} \\
\newpage
\textbf {\large Grade Distribution:} \\
\hspace*{40mm}
\begin{tabular}{ l l }
Attendance and Participation & 10\% \\
Midterm Exam  & 30\% \\
Journal Article Review & 15\% \\
Group Presentation & 15\% \\
Final Exam  & 30\% \\
Extra Credits  & no more than 10\%
\end{tabular} \\\\
% Course Policies. These are just examples, modify to your liking.
\textbf {\large Course Policies:}
\begin{itemize}
	\item \textbf {Attendance and Participation}
		\begin{itemize}
			\item Please attend classes on time and finish required readings before class.
			\item Missing more than \textbf{3} classes will result in grade deductions. If you must miss a class, please email me beforehand.
			CSI’s attendance policy establishes that: “Students are expected to attend all sessions. A student who is absent in excess of 15 percent of the class hours in one semester is assigned a grade of WU (withdrew unofficially), subject to the discretion of the instructor.” 
		\end{itemize}
	\item \textbf {Midterm and Final Exam }
		\begin{itemize}
			\item Midterm and Final exam will be closed book, closed notes, in-class exams. 
			\item Total score for midterm and final exam will be 100'. There will be 25 multiple choice questions, 2' each. Three short essay questions, 10' each. One long essay question, 20'. 
			\item Make-up exam is only available to those who have reasonable excuses. If you have to miss the exam, please contact me \textbf{BEFORE} exam day.
		\end{itemize}
	\item \textbf {Journal Article Review and Group Presentation}
		\begin{itemize}
			\item Each group consists of 2-3 people and the presentation should last around 30 minutes.   
			\item The group presentation should review an article of your choice. Articles will be posted blackboard. Please choose one that interests you most.
			\item Presentation should cover the research question(s), background, methods, data, and conclusion. 
			\item After the presentation, each student is required to submit a 3-page single spaced journal article review on the article they presented.
		\end{itemize}
	\item \textbf{Extra Credits}
		\begin{itemize}
			\item Extra credits will be available in the form of homework and reading responses. Homework and reading responses are optional. Doing those will earn you extra credit.
			\item You can earn up to 10' from extra credits. 
		\end{itemize}
\end{itemize}
\newpage
% College Policies
\textbf {\large College Policies:} 
% This should be specific to your instituition, an example is provided.

\textbf{Accommodations for Students with Disabilities: }

\hspace{3mm}
\hangindent=5mm Qualified students with disabilities will be provided reasonable academic accommodations if determined eligible by the Center for Student Accessibility. Prior to granting disability accommodations in this course, the instructor must receive written verification of student’s eligibility from the Center for Student Accessibility, which is located in 1P-101. It is the student’s responsibility to initiate contact with the Center for Student Accessibility staff and to follow the established procedures for having the accommodation notice sent to the instructor.
To learn more about the accommodations and services that are available, please contact
\\
Center for Student Accessibility \\ 718-982-2510
Center for the Arts (1P), Room 101 \\ CSA@csi.cuny.edu


\textbf{Policies for Dishonorable Conduct/Cheating:}

\hspace{3mm}
\hangindent=5mm 
Academic dishonesty includes cheating and plagiarism. This can consist of (1) copying answers from another student's exam, (2) copying ideas or words from another student's paper, (3) copying ideas or words from a published source, including websites, without appropriate citation, and 4) using a cell phone or other method to get assistance on a quiz or exam. Students who cheat or plagiarize will have consequences ranging from receiving no points on the assignment/test to being suspended from CSI (with acknowledgement that you cheated on your permanent record). In all cases, the Psychology department chair will be alerted about the offense and a formal report will be reported to the Dean of Student Affairs.Please see CUNY’s policy on Academic Integrity: http://web.cuny.edu/academics/info-central/policies/academic-integrity.pdf

\textbf{Other CSI Resources:}
\begin{itemize}
    \item \textbf{The Counseling Center} \\
    The Counseling Center provides individual and group counseling for students of the College of Staten Island. We offer personal and academic counseling services. Students are given the opportunity to explore issues that can help them achieve success. To make an appointment please call 718-982-2391 or drop-in to 1A-109. Students can also be seen on a walk-in basis. \\ Email: counseling@csi.cuny.edu
     \item \textbf{The Writing Center} \\
     2S-216\\
     Bob Brandt, Tutoring Coordinator\\
     Contact Info: 718-982-3635\\
     The Writing Center, under the direction of the English Department, assists students in improving reading and writing skills in all subject areas. Our basic approach is to help students improve their skills by providing them with meaningful feedback and guidance. We are here to help each student thoroughly fulfill his or her own potential through a better understanding of course requirements, assignments and readings.
     To meet these goals, we offer two primary modes of tutoring. Students with regularly scheduled appointments meet for one class period (50 minutes) a week with a designated tutor, sometimes one-on-one and sometimes in small groups that include up to three other students in the same class or at the same level. Drop-in sessions, during which students are seen on a first-come, first-served basis, require no appointment.

\end{itemize}
\end{document}



